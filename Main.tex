\documentclass{beamer}
\usetheme{Madrid}
% \usetheme{Frankfurt}
% \usetheme{Darmstadt}
% \usetheme{Berlin}
% \usetheme{Warsaw}
% \usetheme{Berkeley}
% \usetheme{Bergen}
% \usetheme{CambridgeUS}
% \usetheme{Copenhagen}

\providecommand{\pr}[1]{\ensuremath{\Pr\left(#1\right)}}
\providecommand{\cbrak}[1]{\ensuremath{\left\{#1\right\}}}
\newcommand{\mysolution}{\noindent \textbf{Solution: }}

\title{AI1110 - Probability and Random Variables \\
        Assignment 4}
\author{Aakash Kamuju \\
        ai21btech11001}
        \begin{document}
\maketitle
    % \begin{frame}{Table of Contents}
    % \tableofcontents
    % \end{frame}
    \begin{frame}{CBSE class 12 Chapter 13}
        \begin{section}{Question}
            \begin{block}{Example 10}
             A black and a red dice are rolled.\\
(a) Find the conditional probability of obtaining a sum greater than 9,given that the black die resulted in a 5\\
(b) Find the conditional probability of obtaining the sum 8 , given that the red die resulted in a number less than 4
   \end{block}
    \end{section}
    \end{frame}
    \begin{frame}
        \begin{section}{Solution}
            \begin{block}{Solution}
           (a) Let $X \in \{0,1\} $is random variable that denote whether the sum is greater than 9 or not \\Let X = 0 denotes the sum is less than or equal to 9 and Y = 1 denotes that sum is greater than 9.\\
           
              Events when sum is greater than 9 are (4,6),(5,5),(5,6),(6,4),(6,5),(6,6)
              Total events are $6^2 $ = 36\\
              \pr{X =1} = 6/36
            
            \end{block}
            \end{section}
            \end{frame}
    \begin{frame}
        \begin{section}{Solution Continued}
            \begin{block}{Solution Continued}
             Let $Y \in \{0,1\} $is random variable that denote whether the number on black die is 5 or not \\Let Y = 0 denotes the number on black die is not 5 and X = 1 denotes that number on black die is 5.\\
             Events that satisfy Y are (5,1),(5,2),(5,3),(5,4),(5,5),(5,6)
             Events that satisfy both X,Y are (5,5),(5,6)\\
             \pr{Y = 1,X = 1} = $\frac{2}{36}$
     \end{block}         
    \end{section}
    \end{frame}
    \begin{frame}
        \begin{block}{Solution Continued...}
            The desired probability is
            \begin{align}
                \pr{Y=1 | X=1} &= \frac{\pr{Y=1, X=1}}{\pr{X=1}} \text{ (Bayes' Theorem)}\\
                \pr{Y=1 | X=1} &= \frac{\frac{2}{36}}{\frac{6}{36}} = \frac{1}{3}
            \end{align}
            
        \end{block}
    \end{frame}
     \begin{frame}
        \begin{section}{Solution Continued}
            \begin{block}{Solution Continued}
            (b) Let $X \in \{0,1\} $is random variable that denote whether the sum is 8 or not \\Let X = 0 denotes the sum is not 8 and X = 1 denotes that sum is 8.\\
              Events when sum is 8 are (2,6),(3,5),(4,4),(5,3),(6,2).
              Total events are $6^2 $ = 36\\
              $\pr{X=1} = \frac{5}{36}$
            \end{block}
            \end{section}
            \end{frame}
    \begin{frame}
        \begin{section}{Solution Continued}
            \begin{block}{Solution Continued}
             Let $Y \in \{0,1\} $is random variable that denote whether the number on red die is less than 4 or not \\Let Y = 0 denotes the number on red die is not less than 4 and X = 1 denotes that number on red die less than 4.\\
             Events that satisfy Y are \\
             (1,1),(2,1),(3,1),(4,1),(5,1),(6,1)\\
             (1,2),(2,2),(3,2),(4,2),(5,2),(6,2)\\
             (1,3),(2,3),(3,3),(4,3),(5,3),(6,3)\\
             Events that satisfy both X,Y are (5,3),(6,2)\\
             \pr{Y = 1,X = 1} = $\frac{2}{36}$\\
             \pr{Y = 1} = $\frac{18}{36}$
     \end{block}         
    \end{section}
    \end{frame}
\begin{frame}
        \begin{block}{Solution Continued...}
            The desired probability is
            \begin{align}
                \pr{X=1 | Y=1} &= \frac{\pr{Y=1, X=1}}{\pr{Y=1}} \text{ (Bayes' Theorem)}\\
                \pr{Y=1 | X=1} &= \frac{\frac{2}{36}}{\frac{18}{36}} = \frac{1}{9}
            \end{align}
         
        \end{block}
    \end{frame}
\end{document}
